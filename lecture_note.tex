\documentclass{report}
\usepackage{authblk}
\usepackage{mathptmx}
\usepackage{url,latexsym,amsmath,amsthm,xspace,rotating,multirow,multicol,xspace,amssymb,paralist}
\usepackage{euscript}
\usepackage{fancybox,xcolor}
\usepackage{longtable}
\usepackage{paralist}
\usepackage[normalem]{ulem}
\usepackage[pdftex]{hyperref}
\usepackage{algorithmicx}
\usepackage{algpseudocode}
\usepackage{algorithm}
\usepackage{cancel}

\usepackage{url}
\usepackage{latexsym}

\usepackage{times}
\usepackage{amsmath}
\usepackage{amsthm}
\usepackage{amssymb}
\usepackage{graphicx}
\usepackage{xspace}
\usepackage{tabularx}
\usepackage{multicol}
\usepackage{multirow}
%\usepackage{hyperref}
\usepackage{url}
%\usepackage{natbib}
\usepackage{wrapfig}
\usepackage{comment}
\usepackage{listings}
\usepackage{color}
\usepackage[utf8]{inputenc}
\usepackage{fancyvrb}
\usepackage{booktabs}
\usepackage{color}
\usepackage[normalem]{ulem}

\newcommand{\obs}{\text{obs}}
\newcommand{\mis}{\text{mis}}

\newcommand{\qt}[1]{\left<#1\right>}
\newcommand{\ql}[1]{\left[#1\right]}
\newcommand{\hess}{\mathbf{H}}
\newcommand{\jacob}{\mathbf{J}}
\newcommand{\hl}{HL}
\newcommand{\cost}{\mathcal{L}}
\newcommand{\lout}{\mathbf{r}}
\newcommand{\louti}{r}
\newcommand{\outi}{y}
\newcommand{\out}{\mathbf{y}}
\newcommand{\gauss}{\mathbf{G_N}}
\newcommand{\eye}{\mathbf{I}}
\newcommand{\softmax}{\text{softmax}}
\newcommand{\targ}{\mathbf{t}}
\newcommand{\metric}{\mathbf{G}}
\newcommand{\sample}{\mathbf{z}}
\newcommand{\f}{\text{f}}
%\newcommand{\log}{\text{log}}

\newcommand{\bmx}[0]{\begin{bmatrix}}
\newcommand{\emx}[0]{\end{bmatrix}}
\newcommand{\qexp}[1]{\left<#1\right>}
\newcommand{\vect}[1]{\mathbf{#1}}
\newcommand{\vects}[1]{\boldsymbol{#1}}
\newcommand{\matr}[1]{\mathbf{#1}}
\newcommand{\var}[0]{\operatorname{Var}}
\newcommand{\std}[0]{\operatorname{std}}
\newcommand{\cov}[0]{\operatorname{Cov}}
\newcommand{\diag}[0]{\operatorname{diag}}
\newcommand{\matrs}[1]{\boldsymbol{#1}}
\newcommand{\va}[0]{\vect{a}}
\newcommand{\vb}[0]{\vect{b}}
\newcommand{\vc}[0]{\vect{c}}
\newcommand{\ve}[0]{\vect{e}}

\newcommand{\vh}[0]{\vect{h}}
\newcommand{\vv}[0]{\vect{v}}
\newcommand{\vx}[0]{\vect{x}}
\newcommand{\vp}[0]{\vect{p}}
\newcommand{\vz}[0]{\vect{z}}
\newcommand{\vw}[0]{\vect{w}}
\newcommand{\vs}[0]{\vect{s}}
\newcommand{\vf}[0]{\vect{f}}
\newcommand{\vi}[0]{\vect{i}}
\newcommand{\vo}[0]{\vect{o}}
\newcommand{\vd}[0]{\vect{d}}
\newcommand{\vy}[0]{\vect{y}}
\newcommand{\vg}[0]{\vect{g}}
\newcommand{\vm}[0]{\vect{m}}
\newcommand{\vu}[0]{\vect{u}}
\newcommand{\vL}[0]{\vect{L}}
\newcommand{\vr}[0]{\vect{r}}
\newcommand{\vone}[0]{\vect{1}}

\newcommand{\mW}[0]{\matr{W}}
\newcommand{\mE}[0]{\matr{E}}
\newcommand{\mG}[0]{\matr{G}}
\newcommand{\mX}[0]{\matr{X}}
\newcommand{\mY}[0]{\matr{Y}}
\newcommand{\mQ}[0]{\matr{Q}}
\newcommand{\mU}[0]{\matr{U}}
\newcommand{\mF}[0]{\matr{F}}
\newcommand{\mV}[0]{\matr{V}}
\newcommand{\mA}{\matr{A}}
\newcommand{\mC}{\matr{C}}
\newcommand{\mD}{\matr{D}}
\newcommand{\mL}[0]{\matr{L}}
\newcommand{\mR}[0]{\matr{R}}
\newcommand{\mS}{\matr{S}}
\newcommand{\mI}{\matr{I}}
\newcommand{\td}[0]{\text{d}}
\newcommand{\TT}[0]{\vects{\theta}}
\newcommand{\vsig}[0]{\vects{\sigma}}
\newcommand{\valpha}[0]{\vects{\alpha}}
\newcommand{\vmu}[0]{\vects{\mu}}
\newcommand{\vzero}[0]{\vect{0}}
\newcommand{\tf}[0]{\text{m}}
\newcommand{\tdf}[0]{\text{dm}}
\newcommand{\grad}[0]{\nabla}
\newcommand{\alert}[1]{\textcolor{red}{#1}}
\newcommand{\N}[0]{\mathcal{N}}
\newcommand{\YY}[0]{\mathcal{Y}}
\newcommand{\BB}[0]{\mathcal{B}}
\newcommand{\LL}[0]{\mathcal{L}}
\newcommand{\HH}[0]{\mathcal{H}}
\newcommand{\RR}[0]{\mathbb{R}}
\newcommand{\MM}[0]{\mathcal{M}}
\newcommand{\OO}[0]{\mathbb{O}}
\newcommand{\II}[0]{\mathbb{I}}
\newcommand{\Scal}[0]{\mathcal{S}}
\newcommand{\sigmoid}{\sigma}
\newcommand{\E}[0]{\mathbb{E}}
\newcommand{\enabla}[0]{\ensuremath{%
    \overset{\raisebox{-0.3ex}[0.5ex][0ex]{%
    \ensuremath{\scriptscriptstyle e}}}{\nabla}}}
\newcommand{\enhnabla}[0]{\nabla_{\hspace{-0.5mm}e}\,}
\newcommand{\eos}[0]{\ensuremath{\left< \text{eos}\right>}}


\newcommand{\todo}[1]{{\Large\textcolor{red}{#1}}}
\newcommand{\done}[1]{{\Large\textcolor{green}{#1}}}
\newcommand{\dd}[1]{\ensuremath{\mbox{d}#1}}

\DeclareMathOperator*{\argmax}{\arg \max}
\DeclareMathOperator*{\argmin}{\arg \min}
\newcommand{\newln}{\\&\quad\quad{}}

\newcommand{\BP}{\text{BP}}
\newcommand{\PPL}{\text{PPL}}
\newcommand{\PL}{\text{PL}}
\newcommand{\MatSum}{\text{MatSum}}
\newcommand{\MatMul}{\text{MatMul}}
\newcommand{\KL}{\text{KL}}
\newcommand{\data}{\text{data}}
\newcommand{\rect}{\text{rect}}
\newcommand{\maxout}{\text{maxout}}
\newcommand{\train}{\text{train}}
\newcommand{\val}{\text{val}}
\newcommand{\init}{\text{init}}
\newcommand{\fenc}{\text{fenc}}
\newcommand{\renc}{\text{renc}}
\newcommand{\enc}{\text{enc}}
\newcommand{\dec}{\text{dec}}
\newcommand{\test}{\text{test}}
\newcommand{\Ax}{\mathcal{A}_x}
\newcommand{\Ay}{\mathcal{A}_y}
\newcommand{\ola}{\overleftarrow}
\newcommand{\ora}{\overrightarrow}
\newcommand{\ov}{\overline}
\newcommand{\ts}{\rule{0pt}{2.6ex}}       % Top strut
\newcommand{\ms}{\rule{0pt}{0ex}}         % Middle strut
\newcommand{\bs}{\rule[-1.2ex]{0pt}{0pt}} % Bottom strut
\newcommand{\specialcell}[2][c]{%
  \begin{tabular}[#1]{@{}c@{}}#2\end{tabular}}


%\usepackage{bibentry}
%\nobibliography*

\begin{document}

\title{Introduction to Machine Learning}
\author{Kyunghyun Cho}
\affil{
    Courant Institute of Mathematical Sciences and \\
    Center for Data Science,\\
    New York University 
}

\maketitle
\pagenumbering{arabic}

%\abstract{
%    This is a lecture note for the course DS-GA 3001 $\left<\right.$Natural
%    Language Understanding with Distributed Representation$\left.\right>$ at the
%    Center for Data Science\footnote{
%        \url{http://cds.nyu.edu/}
%    }, New York University in Fall, 2015. As the name of the course suggests,
%    this lecture note introduces readers to a neural network based approach to
%    natural language understanding/processing. In order to make it as
%    self-contained as possible, I spend much time on describing basics of
%    machine learning and neural networks, only after which how they are used for
%    natural languages is introduced. On the language front, I almost solely
%    focus on language modelling and machine translation, two of which I
%    personally find most fascinating and most fundamental to natural language
%    understanding.
%    
%    After about a month of lectures and about 40 pages of writing this lecture
%    note, I found this fascinating note \cite{goldberg2015primer} by Yoav
%    Goldberg on neural network models for natural language processing. This note
%    deals with wider topics on natural language processing with
%    distributed representations in more details, and I highly recommend you to read it
%    (hopefully along with this lecture note.) I seriously wish Yoav had written
%    it earlier so that I could've simply used his excellent note for my course.
%
%    This lecture note had been written quite hastily as the course progressed,
%    meaning that I could spare only about 100 hours in total for this note.
%    This is my lame excuse for likely many mistakes in this lecture note, and I
%    kindly ask for your understanding in advance. Again, how grateful I
%    would've been had I found Yoav's note earlier.
%
%    I am planning to update this lecture note gradually over time, hoping that I
%    will be able to convince the Center for Data Science to let me teach the
%    same course next year. The latest version will always be available both in
%    pdf and in latex source code from
%    \url{https://github.com/nyu-dl/NLP_DL_Lecture_Note}. The arXiv version will
%    be updated whenever a major revision is made.
%
%    I thank all the students and non-students who took\footnote{
%        In fact, they are still taking the course as of 24 Nov 2015. They have
%        two guest lectures and a final exam left until the end of the course.
%    }
%    this course and David Rosenberg for feedback.
%}
%
%\tableofcontents

%The following text books are recommended in addition to this lecture note:
%\begin{itemize}
%    \item ``Pattern Recognition and Machine Learning'' by Chris Bishop \cite{bishop2006pattern}
%    \item ``Machine Learning: a probabilistic perspective'' by Kevin Murphy \cite{murphy2012machine}
%\end{itemize}

\chapter{Supervised Learning}
\label{chap:supervised}

\section{Classification}
\label{sec:classification}

\subsection{Problem Setup: Logistic Regression}

\subsection{Optimization vs. Learning}

\subsubsection{Expected vs. Empirical Loss Functions}

\subsubsection{Gradient-based Optimization}

\subsection{Support Vector Machines}

\subsubsection{One Problem, Many Loss Functions}

\subsubsection{Nonlinear Feature Function and Kernel Trick}

\subsection{Decision Tree$^*$}

\subsection{Ensemble Methods$^*$}


\section{Regression}
\label{sec:regression}

\subsection{Linear Regression and Regularization}

\subsubsection{Linear Regression}

\subsubsection{Regularization and Prior Distributions}

\subsection{Gaussian Process Regression: Parameter-Space View}

\subsubsection{Bayesian Approach to Machine Learning}

\subsubsection{Gaussian Process Regression}


\chapter{Unsupervised Learning}
\label{chap:unsupervised}

\section{Dimensionality Reduction}
\label{sec:dimred}

\subsection{Problem Setup}

\subsection{Principal Component Anslysis}

\subsubsection{Maximum Variance Criterion}

\subsubsection{Minimum Reconstruction Criterion}

\subsubsection{Probabilistic Principal Component Anlysis}

\paragraph{Expectation-Maximization Algorithm}

\subsubsection{PCA with Missing Values: Collaborative Filtering}

\subsection{Other Dimensionality Reduction Techniques}


\section{Clustering}
\label{sec:cluster}

\subsection{Problem Setup}

\paragraph{Clustering vs. Dimensionality Reduction}

\subsection{$k$-Means Clustering}

\subsection{Mixture of Gaussians}

\subsection{Other Clustering Methods}

\section{Time-Series Modelling}
\label{sec:timeseries}










\bibliographystyle{abbrv}
\bibliography{lecture_note}


\end{document}






